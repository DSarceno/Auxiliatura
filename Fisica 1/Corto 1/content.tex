\begin{multicols}{2}
\begin{enumerate}
	
	\item Al cambio de velocidad que experimenta un cuerpo por cada unidad de tiempo, se le llama:
	\begin{enumerate}[a)]
		\item Desplazamiento
		\item Intervalo de Tiempo
		\item \colorbox[rgb]{1,1,0}{Aceleración}
		\item Rapidez
	\end{enumerate}
	
	
	
	\item Si $A$ tiene dimensionales de $m/s^3$, ¿Cuál de las siguientes expresiones es dimensionalmente correcta?
	\begin{enumerate}[a)]
		\item $x = vA$
		\item $x = Axt/(v/t^2)$
		\item $v = xA/(a/t)$
		\item $v=At^2 /v^3$
		\item \colorbox[rgb]{1,1,0}{NAC}
	\end{enumerate}
	
	
	
	\item La fórmula de $2d/t^2$ es de:
	\begin{enumerate}[a)]
		\item Velocidad Media
		\item \colorbox[rgb]{1,1,0}{Aceleración}
		\item Fuerza
	\end{enumerate}
	
	
	
	\item En MRU: ¿Qué representa la pendiente de la gráfica posición-tiempo?
	\begin{enumerate}[a)]
		\item Desplazamiento total
		\item \colorbox[rgb]{1,1,0}{Velocidad}
		\item Aceleración
		\item NAC
	\end{enumerate}
	
	
	
	\item Una partícula lanzada verticalmente hacia arriba y termina su movimiento por debajo del punto de partida. La coordenada cero se encuentra en el punto de partida, por lo tanto, el desplazamiento será:
	\begin{enumerate}[a)]
		\item Positivo
		\item Cero
		\item \colorbox[rgb]{1,1,0}{Negativo}
	\end{enumerate}
	
	
	
	
	
	
	
	\item Un automóvil viaja al oeste. Tiene aceleración hacia el este. Por consiguiente, la velocidad:
	\begin{enumerate}[a)]
		\item Constante
		\item Cero
		\item \colorbox[rgb]{1,1,0}{Disminuye}
		\item Aumenta
		\item NAC
	\end{enumerate}
	
	
	
	\item Se lanza un objeto verticalmente hacia arriba, en el punto más alto de su trayectoria, la magnitud de la aceleración es:
	\begin{enumerate}[a)]
		\item \colorbox[rgb]{1,1,0}{$g$}
		\item Cero
		\item Igual a $v$
		\item Indeterminada
		\item NAC
	\end{enumerate}
	
	
	
	\item El conductor de un vehículo que circula por una caller recta frena bruscamente para no atropellar a un peatón y recorre $50m$ hasta inmovilizar el coche; si se supone que con la brusca frenada consigue una deceleración de $16m/s^2$, ¿a qué velocidad circulaba antes de frenar?
	\begin{enumerate}[a)]
		\item $40km$
		\item $72km$
		\item $101.52km$
		\item \colorbox[rgb]{1,1,0}{$144km$}
		\item NAC
	\end{enumerate}
	
	
	
	\item Considere los siguientes controlesen un automóvil: acelerador, freno, volante. ¿En ésta lista cuáles son los controles que provocan una aceleración en el automóvil?
	\begin{enumerate}[a)]
		\item Freno
		\item Volante
		\item Acelerador
		\item A y C
		\item \colorbox[rgb]{1,1,0}{Todos}
		\item NAC
	\end{enumerate}
	
	
	
	\item Desde lo alto de un rascacielos se lanzan un lingote de Au y una pelota de polietileno verticalmente hacia arriba con la misma velocidad incial. Ignorando la resistencia del aire. ¿Cuál de los dos alcanzará primero el suelo?
	\begin{enumerate}[a)]
		\item Lingote de Au
		\item Pelota de polietileno
		\item Faltan datos
		\item \colorbox[rgb]{1,1,0}{NAC}
	\end{enumerate}
	
\end{enumerate}
\end{multicols}

































%%%%