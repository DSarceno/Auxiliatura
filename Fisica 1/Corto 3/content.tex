\section*{Preguntas}
\begin{multicols}{2}
\begin{enumerate}
	
	\item Qué representa el área bajo la curva en una gráfica Fuerza-Posición?
	\begin{enumerate}[a)]
		\item Desplazamiento
		\item Aceleración
		\item \colorbox[rgb]{1,1,0}{Trabajo}
		\item NAC
	\end{enumerate}
	
	
	
	\item Cuando un objeto se mueve hacia arriba, el trabajo realizado por la fuerza de la gravedad es:
	\begin{enumerate}[a)]
		\item Cero
		\item \colorbox[rgb]{1,1,0}{Negativo}
		\item Positivo
		\item NAC
	\end{enumerate}
	
	\columnbreak
	
	\item Una nave espacial se mueve alrededor de la Tierra en una órbita circular con un radio constante. ¿Cuánto trabajo aplica la fuerza gravitatoria en la nave espacial durante una vuelta?
	\begin{enumerate}[a)]
		\item $F_g d$
		\item $-F_g d$
		\item $mgr$ (con $r$ el radio de la órbita)
		\item \colorbox[rgb]{1,1,0}{Cero}
		\item $\frac{1}{2} mv^2$
	\end{enumerate}

	
	
	\item ¿Qué sucede con la energía total de un objeto en movimiento si se conservan todas las fuerzas aplicadas?
	\begin{enumerate}[a)]
		\item Aumenta
		\item Disminuye
		\item \colorbox[rgb]{1,1,0}{Permanece Constante}
		\item Faltan datos
		\item NAC
	\end{enumerate}
	
	
	
	\item Por qué la fuerza de un resorte es negativa? $\mathbf{R//}$ Porque es una fuerza de restitución.
	
	
	
\end{enumerate}
\end{multicols}



\pagebreak

\section*{Problema}
Un bloque pequeño con masa de $0.04kg$ se desliza en un círculo vertical de radio $0.5m$ en el interior de una pista circular. En una de las revoluciones, cuando el bloque se encuentra en la parte inferior de la trayectoria, el punto $A$, la magnitud de la fuerza normal ejercida por la pista sobre el bloque es $3.95N$. En la misma revolución, cuando el bloque alcanza la parte superior de su trayectoria, el punto $B$, la manitud de la fuerza normal es de $0.068N$. ¿Cuánto trabajo realiza la fricción sobre el bloque cuando este se desplaza del punto $A$ al punto $B$?


\vspace{1cm}

\textbf{Respuesta: } \colorbox[rgb]{1,1,0}{$0.3825J$}























%%%%