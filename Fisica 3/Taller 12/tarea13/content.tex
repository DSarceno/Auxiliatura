\section*{Problema}
Se sabe que para un conductor recto el campo es
	$$ B = \frac{\mu _o I}{2\pi r}. $$
\begin{description}
	\item[A:] Por la regla de la mano derecha se tiene el siguiente valor para el campo en $A$
		$$ B_A = B_1 \cos{\pi /4} + B_2 \cos{\pi /4} + B_3, $$
			con $B_1 = B_2$
				$$ B_A = 2\qty(\frac{\mu _o I}{2\pi a\sqrt{2}}) \cos{\pi /4} + \frac{\mu _o I}{2\pi (3a)} = \boxed{ 53.3\mu T \, \downarrow . } $$
	\item[B:] Dado que $B_1 = -B_2$, se tiene
		$$ B_B = B_3 = \frac{\mu _o I}{2\pi (2a)} = \boxed{ 20\mu T \, \downarrow . } $$
	\item[C:] Este caso es parecido a $A$, pero las componentes de $B_1$ y $B_2$ van hacia arriba, entonces
		$$ B_C = 2\qty(\frac{\mu _o I}{2\pi a\sqrt{2}}) \cos{\pi /4} - \frac{\mu _o I}{2\pi a} = \boxed{ 0T. } $$
\end{description}




%%%%