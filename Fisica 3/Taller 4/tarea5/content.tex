\section*{Problema}

\begin{enumerate}[a)]
	\item Como se trata de un conductor, el campo dentro de él es cero. Mientras que el voltaje tiene un valor de
		$$ V = \frac{q}{4\pi \varepsilon _o r} = 1.67\times 10^{6} V. $$
	\item Fuera de la esfera el campo se comporta como el de una carga puntual, por ende
		$$ E(0.2) = 5.8\times 10^{6} N/C, $$
		$$ V(0.2) = 1.168\times 10^{6} V. $$
	\item Al igual que el anterior inciso, es equivalente a una carga puntual.
		$$ E(0.14) = 11.9\times 10^{6} N/C, $$
		$$ V(0.14) = 1.67\times 10^{6} V. $$
\end{enumerate}










%%%%