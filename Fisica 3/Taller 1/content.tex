\section*{Problema 1}
Sabiendo que el número atómico de la plata es $47$, se tienen esa cantidad de electrones. Entonces, utilizando el número de avogadro\footnote{Es el factor de proporcionalidad que relaciona el número de partículas en una muestra con la cantidad de sustancia de la misma.} y con un poco de aritmética se llega a
	$$ N = \qty(\frac{10g}{107.87g/\text{mol}}) \qty(6.022\times 10^{23} \text{atomos}/\text{mol}) \qty(47 \text{electrones}/\text{atomo}) = 2.62\times 10^{24} \text{electrones}. $$
	
Ahora, encontramos el número de electrones en $1mC$
	$$ \frac{Q}{e} = \frac{1mC}{1.6\times 10^{15}} = 6.24\times 10^{15} \text{electrones}, $$

que con un poco de aritmética, se encuentra que hay $2.38$ electrones agregados por cada $10^9$ existentes.