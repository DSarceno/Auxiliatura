\section*{Problema}

\begin{enumerate}[a)]
	\item Calculamos la corriente $I = \frac{\varepsilon}{R + r}$ y, con esto, la diferencia de potencial.
		$$ \boxed{ \Delta V = IR = 12.4V } $$
	\item Se tiene que $I_{\text{bateria}} = I_{\text{luces}} + I_{\text{carro}}$, por la batería $\varepsilon = I_{\text{bateria}} r + I_{\text{luces}} R$, sustituyendo la corriente de la batería encontramos la corriente por las luces $I_{\text{luces}} = 1.93A$. Con esto, encontramos la diferencia de potencial
		$$ \boxed{\Delta V = 1.93A*5\Omega = 9.65V .} $$
\end{enumerate}









%%%%