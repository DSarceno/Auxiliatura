\begin{mdframed}[style=warning]
	\begin{ejercicio}
		Considere un cilindro con una pared delgada uniformemente cargada con una carga total $Q$, radio $R$ y altura $h$. Determine el campo eléctrico en un punto a una distancia $d$ del centro del lado derecho del cilindro. Realice lo anterior para un cilindro sólido.
	\end{ejercicio}
\end{mdframed}






\begin{mdframed}[style=warning]
	\begin{ejercicio}
		Una carga negativa $-q$ esta situada en el centro de un anillo con carga uniforme, que tiene una carga positiva total $Q$ y radio $a$. La partícula, limitada a moverse a lo largo del eje $x$, es desplazada una pequeña distancia $x$, con $x\ll a$, y luego se libera. Demuestre que la partícula oscila en un movimiento armónico simple con una frecuencia igual a
		$$ \nu = \frac{1}{2\pi} \qty(\frac{k_e qQ}{ma^3}) ^{\flatfrac{1}{2}} .$$
	\end{ejercicio}
\end{mdframed}






\begin{mdframed}[style=warning]
	\begin{ejercicio}
		Dos varillas delgadas de longitud $L$ están a lo largo del eje x, una entre $x = a/2$ y $x = a/2 + L$, y la otra entre $x = -a/2$ y $x = -a/2 - L$. Cada una tiene carga positiva $Q$ distribuida de manera uniforme.
		\begin{enumerate}[a)]
			\item Calcule el campo eléctrico producido por la segunda varilla en puntos a lo largo del eje $x$ positivo.
			\item Demuestre que la magnitud de la fuerza que ejerce una varilla sobre la otra es
				$$ F = \frac{Q^2}{4\pi \varepsilon _o L^2} \ln{\qty[\frac{(a + L)^2}{a(a + 2L)}]}. $$
			\item Demuestre que si $a \gg L$, la magnitud de esta fuerza se reduce a $F = \flatfrac{Q^2}{4\pi \varepsilon _o a^2}$. Interprete el resultado.
		\end{enumerate}
	\end{ejercicio}
\end{mdframed}



































%%%