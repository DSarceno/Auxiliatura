% AUTOR: Diego Sarceño

% ENCABEZADO DE TRABAJOS CON LOGO DE LA UNIDAD ACADÉMICA

% ENCABEZADO LOGO COLOR
%\begin{tabulary}{20cm}{Lp{0.9cm}p{6.1cm}}
%Universidad de San Carlos de Guatemala & & \multirow{4}{8cm}{\hfill %\includegraphics[scale=0.5]{ECFM.png}}\\            % Logo de la unidad academica
%Escuela de Ciencias Físicas y Matemáticas & \hfill & \\
%Diego Sarceño 201900109 & \hfill & \\
%Análisis de Variable Compleja 1 & \hfill & \\
%\today & & \\
%\end{tabulary}\\[0.25cm]


% ENCABEZADO LOGOS
%\noindent 
%\begin{tabulary}{20cm}{LLCRR}
%\multirow{4}{2.3cm}{\includegraphics[scale=0.13]{/home/diego/Documents/Licenciatura/LatexBasic/ECFM.pdf}} & Universidad de San Carlos de Guatemala  &  & ~\hfill & \multirow{4}{4.3cm}{\hfill \includegraphics[scale=0.082]{/home/diego/Documents/Licenciatura/LatexBasic/USAC.pdf}}\tabularnewline
% & Escuela de Ciencias Físicas y Matemáticas &  &  & \tabularnewline
% & Métodos Matemáticos para Física, Semestre 2, 2023 & &   & \tabularnewline
% & Profesor: Dr. Juan Ponciano & &  & \tabularnewline
% & Auxiliar: Diego Sarceño &  & & \tabularnewline
%\end{tabulary}\\[0.75cm]

%{\hrule height 1.5pt} \vspace{0.1cm}
%\begin{tabulary}{21cm}{p{5.5cm}p{11.5cm}p{2cm}}
%    \hfill & \huge{\scshape{Guía 3}} & \hfill
%\end{tabulary}
%{\hrule height 1.5pt} 
%\vspace{0.5cm}


%{\hrule height 1.5pt}
%\begin{center}
%	\huge{\scshape{Guía 5}}
%\end{center}
%{\hrule height 1.5pt} 






\textcolor{DS_Black}{
\begin{minipage}{0.85\textwidth}
    \begin{center}
        \textbf{\Large Set Problemas 1}\\
        \vspace{5pt}
        Preparación Olimpiada Iberoamericana de Física \\
        \vspace{20pt}
        \today
    \end{center}
\end{minipage}
\vspace{10pt}
\hrule
}