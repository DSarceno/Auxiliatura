\section*{Problema 14.57, Z}

\noindent Una onda transversal qeu viaja en una cuerda tiene amplitud de $0.3 cm$, longitud de onda de $12cm$ y rapidez de $6cm/s$ y se representa mediante 
	$$ y(x,t) = A\cos{\qty[\frac{2\pi}{\lambda} \qty(x - vt)]}. $$
\noindent $a)$ En el tiempo $t = 0$, calcule $y$ a intervalos de $x$ de $1.5cm$ desde $x = 0$ y $x = 12cm$. Muestre los resultados en una gráfica. Esta es la forma de la cuerda en el tiempo $t = 0$. $b)$ REpita los cálculos para los mismos valores de $x$ en $t = 0.4s$ y $t = 0.8s$. Muestre gráficamente la forma de la cuerda en estos instantes. ¿En qué dirección viaja la onda? Escriba los datos en una tabla.







%%%%