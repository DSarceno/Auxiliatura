\begin{mdframed}[style=warning]
	\textbf{Conceptos}
		\begin{enumerate}
			\item Dé una descripción de lo que pasa físicamente cuando empieza a fluir el aire en el tubo de un organo hasta que el sonido se vuelve estable.
			\item Cuando el sonido pasa de aire a agua, ¿cuál de las siguientes cantidades se mantiene constante: longitu de onda, velocidad de onda, frecuencia, amplitude de desplazamiento de las moléculas que propagan el sonido?
			\item Explique brevemente la diferencia entre $dB$, $dB(SPL)$, $dB(A)$ y $dBm$.
		\end{enumerate}
\end{mdframed}



















\begin{mdframed}[style=warning]
	\begin{ejercicio}
		Suponga que un tubo de longitud $L$ contiene un gas y que usted desea tomar la temperatura de ese gas, pero sin introducirse en el tubo. Un extremo está cerrado y el otro está abierto, y un pequeño altavoz que produce sonido de frecuencia variable se encuentra en el extremo abierto. Usted aumenta gradualmente la frecuencia del altavoz hasta qeu el sonido del tubo se vuelve muy ruidoso. Con un aumento posterior de la frecuencia, la intensidad disminuye, pero el sonido vuelve a ser muy intenso otra vez a frecuencias todavía más altas. Sea $f_o$ la frecuencia más baja a la que el sonido es muy intenso. $a)$ demuestre que la temperatura absoluta de este gas está dada por 
			$$ T = \frac{16ML^2 f_o ^2}{\gamma R}, $$
		donde $M$ es la masa molar, $\gamma$ es la razón de sus capacidades calorífica y $R$ es la constante de gas ideal. $b)$ ¿A qué frecuencia por arriba de $f_o$ el sonido del tuvo alcanzará su volumen máximo? $c)$ ¿Cómo podría determinarse la rapidez del sonido en este tuvo a temperatura $T$?
	\end{ejercicio}
\end{mdframed}


























\begin{mdframed}[style=warning]
	\begin{ejercicio}
		Una fuente de sonido se mueve por el eje $x$, entre los detectores $A$ y $B$. La longitud de onda del sonido detectado en $A$ es $0.5$ veces la detectada en $B$. ¿Cuál es la razón $v_s /v$ de la velocidad de la fuente con la del sonido?
	\end{ejercicio}
\end{mdframed}









































%%%