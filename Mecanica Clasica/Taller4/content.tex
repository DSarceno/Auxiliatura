\begin{mdframed}[style=warning]
	\begin{ejercicio}
		Encuentre la posición y velocidad de una partícula, como funciones del tiempo, que se mueve horizontalmente en un medio viscoso cuya fuerza de retardo es proporcional a la velocidad instantanea de la partícula. Utilice los siguientes valores $m = 1kg$, $v_o = 10m/s$, $x_o = 0$, y $k = 0.1s^{-1}$. Realize las gráficas $v-t$, $x-t$ y $v,x$.
	\end{ejercicio}
\end{mdframed}

\begin{mdframed}[style=warning]
	\begin{ejercicio}
		Un proyectil es lanzado desde el origen de un sistema de coordenadas con velocidad $v_o$ con un ángulo $\alpha$ sobre la horizontal, calcule el tiempo que requiere el proyectil para cruzar por una línea que pasa a travéz del origen y hace un ángulo $\beta < \alpha$ con la horizontal.
	\end{ejercicio}
\end{mdframed}

\begin{mdframed}[style=warning]
	\begin{ejercicio}
		Considere un proyectil lanzado verticalmente en un campo gravitacional constante. Para las mismas velocidades iniciales, compare los tiempos que le toma al mismo proyectil alcanzar su altura máxima \textbf{(a)} para fuerza de resistencia cero, \textbf{(b)} para una resistencia proporcional a la velocidad instantanea del proyectil.
	\end{ejercicio}
\end{mdframed}

\begin{mdframed}[style=warning]
	\begin{ejercicio}
		Un proyectil es lanzado con velocidad inicial $v_o$ con un ángulo de elevación $\alpha$ sobre una colina de inclinación $\beta$ ($\beta < \alpha$). Encuentre \textbf{(a)} qué tan lejos caera el proyectil sobre la colina? \textbf{(b)} A qué ángulo $\alpha$ el rango es máximo? \textbf{(c)} Cuál es rango máximo?
	\end{ejercicio}
\end{mdframed}
