\begin{mdframed}[style=warning]
	\begin{ejercicio}
		Una partícula de masa $m$ en reposo en un tiempo $t = 0$ esta sujeta a una fuerza $F(t) = F_o \sin ^2 {\omega t}$. Encuentre las soluciónes $x(t)$ y $v(t)$ y grafiquelas para diferentes frecuencias angulares.
	\end{ejercicio}
\end{mdframed}




\begin{mdframed}[style=warning]
	\begin{ejercicio}
		Una partícula de masa $m$ es repelida del origen por una fuerza inversamente proporcional al cubo de la distancia desde el origen. Plantee y resuelva la ecuación de movimiento si la partícula es inicialmente en reposo a una distancia $x_o$ del origen.
	\end{ejercicio}
\end{mdframed}




\begin{mdframed}[style=warning]
	\begin{ejercicio}
		Una partícula de masa $m$ esta sujeta a una fuerza cuyo potencial es
			$$V(x) = ax^2 - bx^3.$$
		Encuentre la fuerza. La partícula inicia en el origen con velocidad inicial $v_o$, muestre que si $\abs{v_o} < v_c$, donde $v_c$ es una velocidad crítica, la partícula se mantedrá confinada a una región cercana al origen. Encuentre $v_c$.
	\end{ejercicio}
\end{mdframed}




\begin{mdframed}[style=warning]
	\begin{ejercicio}
		De acuerdo a la teoría de Yukawa de fuerzas nucleares, la fuerza de atracción entre un neutrón y un protón tiene el potencial
			$$ V(r) = \frac{K e^{-ar}}{r}, \qquad K < 0. $$
		Encuentre la fuerza, comparela con una ley del cuadrado inverso y grafiquelas.
	\end{ejercicio}
\end{mdframed}




\begin{mdframed}[style=warning]
	\begin{ejercicio}
		Resuelva:
		\begin{itemize}
			\item Las fuerzas conservativas tienen una gran importancia en la física, explique ¿Por qué? ¿Qué implicaciones físicas tiene el hecho de que una fuerza sea conservativa?
			\item ¿Cuáles de las siguientes fuerzas son conservativas? Si lo son, encuentre $V(r)$ y grafíquelas.
			\begin{itemize}
				\item $F_x = ayz + bx + c$, $F_y = axz + bz$, $F_z = axy + by$
				\item $F_x = -ze^{-x}$, $F_y = \ln{z}$, $F_z = e^{-x} + \flatfrac{y}{z}$
				\item $F_\rho = a\rho ^2 \cos{\varphi}$, $F_\varphi = a\rho ^2 \sin{\varphi}$, $F_z = 2az^2$
			\end{itemize}
		\end{itemize}
	\end{ejercicio}
\end{mdframed}

