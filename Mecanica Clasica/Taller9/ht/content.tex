\begin{mdframed}[style=warning]
	\begin{ejercicio}
		Discuta el movimiento de un oscilador amortiguado si la resistencia de amortiguamiento es negativa ($b < 0$).
	\end{ejercicio}
\end{mdframed}




\begin{mdframed}[style=warning]
	\begin{ejercicio}
		Un resorte de masa despresiable cuelga del techo. Una masa se ata en la parte inferior del resorte y se suelta el sistema. ¿Qué tan cerca estará la masa del putno de equilibrio luego de $1$ segundo dado que llega al reposo $0.5m$ debajo del punto del que se soltó y que su movimiento es críticamente amortiguado?
	\end{ejercicio}
\end{mdframed}




\begin{mdframed}[style=warning]
	\begin{ejercicio}
		Un oscilador amortiguado con $\beta < \omega _o$. Se define $\tau$ como el tiempo entre los máximos de $x(t)$. \textbf{(a)} Haga una gráfica de $x(t)$ e indique dónde se muestra $\tau$. Muestre que $\tau = \flatfrac{2\pi}{\omega _1}$. \textbf{(b)} Muestre que una definición equivalente de $\tau$ es que es el doble del tiempo entre ceros consecutivos de $x(t)$. \textbf{(c)} Si $\beta = \flatfrac{\omega _o}{2}$, ¿por qué factor la amplitud disminuye en un periodo?
	\end{ejercicio}
\end{mdframed}





\begin{mdframed}[style=warning]
	\begin{ejercicio}
		Utilice su sofware favorito para graficar el diagrama de fase del oscilador armónico críticamente amortiguado. Demuestre que la ecuación de la línea a la que las trayectorias de fase se acercan de manera asintótica es $\dot{x} = -\beta x$. Grafique al menos $6$ trayectorias de fase.
	\end{ejercicio}
\end{mdframed}





\begin{mdframed}[style=warning]
	\begin{ejercicio}
		La posición $x(t)$  de un oscilador sobreamortiguado está dada por
			$$ x(t) = C_1 e^{-\qty(\beta - \sqrt{\beta ^2 - \omega _o ^2})} + C_2 e^{-\qty(\beta + \sqrt{\beta ^2 - \omega _o ^2})}. $$
		\textbf{(a)} Encuentre las constantes $C_1$ y $C_2$ en términos de la posición inicial $x_o$ y la velocidad $v_o$. \textbf{(b)} Grafíque el comportamiento de $x(t)$ respecto al tiempo para ambos casos $x_o = 0$ y $v_o = 0$. \textbf{(c)} Muestre que si $\beta \to 0$ la solución del inciso \textit{(a)} se aproxima la solución para un movimiento sin amortiguación.
	\end{ejercicio}
\end{mdframed}



















%%

