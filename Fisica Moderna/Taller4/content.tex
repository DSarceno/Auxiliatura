\section*{Tarea 4}
\begin{mdframed}[style=warning]
	\begin{ejercicio}
		Teniendo la velocidad de fase $v_p = \sqrt{\frac{g\lambda}{2\pi}}$, reescribiendola en términos de $\omega$
			$$ v_p = \sqrt{\frac{g}{k}} = \frac{\omega}{k} \qquad \Rightarrow \qquad \omega = \sqrt{gk}. $$
		Entonces, sabiendo que la velocidad de grupo esta dada por $v_g = \dv{\omega}{k}$, se tiene
			$$ v_g = \frac{1}{2} \sqrt{\frac{g}{k}} = \frac{1}{2} v_p. $$
	\end{ejercicio}
\end{mdframed}


\begin{mdframed}[style=warning]
	\begin{ejercicio}
		Dado que la energía del haz es mucho menor que su energía en reposo, podemos utilizar a $\gamma = 1$. Entonces, dada la longitud de onda de De Broglie $\lambda = \frac{h}{mv}$ con $p = mv = \sqrt{2m\text{KE}}$, entonces, sustituyendo en la ecuación de Bragg para $n = 1$
			$$ \phi = \arcsin{\frac{h}{2d\sqrt{2m\text{KE}}}} = 18.65^o. $$
	\end{ejercicio}
\end{mdframed}
