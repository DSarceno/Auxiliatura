\section*{Tarea 3}
\begin{mdframed}[style=warning]
	\begin{ejercicio}
		Solución:
		\begin{enumerate}[a)]
			\item Dada la ecuación del efecto compton,
				$$ \lambda ' - \lambda = \frac{h}{mc} (1-\cos{\theta}), $$
				dada $\lambda = 80pm$, y $120^o$, se tiene que $\lambda ' = 83.64pm $.
			\item Por conservarición del momento, 
				$$ \left\{\begin{array}{c}
						\frac{h\nu}{c} = \frac{h\nu '}{c} \cos{\phi} + p\cos{\theta} \\
						0 = \frac{h\nu '}{c} \sin{\phi} - p\sin{\theta}.
					\end{array}\right. $$
				Resolviendo el sistema para $\theta$, encontramos
					$$ \theta = \tan ^{-1} {\qty( \frac{\frac{1}{\lambda '} \sin{\phi}}{\frac{1}{\lambda} - \frac{1}{\lambda '} \cos{\phi}} )} = 29.26^o .$$
			\item Sabiendo que $KE = hc (\frac{1}{\lambda} - \frac{1}{\lambda '})$, sustituyendo valores: $KE = 674.39eV$.
		\end{enumerate}
	\end{ejercicio}
\end{mdframed}


\begin{mdframed}[style=warning]
	\begin{ejercicio}
		Se sabe que la energía del fotón es $pc$, la masa es $2m + M$ y la energía en reposo es $Mc^2$, entonces, se tiene la siguiente desigualdad
			$$ pc + Mc^2 \geq E, $$
		entonces
			$$ (pc + Mc^2)^2 \geq (pc)^2 + ((2m + M)c^2)^2, $$
		simplificando
			$$ 2pcMc^2 \geq 4mMc^4 + 4m^2 c^4, $$
			$$ pc \geq 2mc^2 \qty(1 + \frac{m}{M}). $$
		para $M\gg m$, entonces, la energía mínima del fotón es $2mc^2 = 1.02MeV$.
	\end{ejercicio}
\end{mdframed}
