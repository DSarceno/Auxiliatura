\section{Tarea 1}
Dado que el tiempo de vida de la partícula esta medido respecto al sistema de referencia en reposos, es el tiempo propio. Por lo tanto, el tiempo medido desde un sistema externo (laboratorio), está dado por la dilatación del tiempo
	\begin{equation}
		t = \frac{t_o}{\sqrt{1 + \frac{v^2}{c^2}}} = \gamma t_o . \label{dilatacion_tiempo}
	\end{equation}
Teniendo $t_o = 1.00 \times 10^{-7} s$ y $v = 0.99c$, entonces la distancia recorrida por la partícula antes de decaer es
	\begin{equation}
		d = vt = \frac{v t_o}{\sqrt{1 + \frac{v^2}{c^2}}} = \boxed{ 210.54 m }
	\end{equation}