%\vspace{1cm}
\section*{Ecuación de Laplace}
Primero recordaremos lo que representan los diferentes operadores diferenciales:


\begin{mdframed}[style=warning]
	{\Large \textbf{Operadores Diferenciales:}} \\
	\begin{description}
		\item[Gradiente $\grad$: ] Es el vector que mide la dirección de "crecimiento" del campo escalar.
			\begin{table}[H]
				\centering
				\caption{Forma del gradiente en diferentes sistemas de coordenadas.}
				\label{grad}
				\begin{tabular}{||c|c||}
					\hline
					\hline
						Coordenadas Cartesianas & $ \grad{\phi} = \displaystyle\pdv{\phi}{x} \vx +\displaystyle\pdv{\phi}{y} \vy + \displaystyle\pdv{\phi}{z} \vz $ \\
					\hline
						Coordenadas Cilíndricas & $ \grad{\phi} = \displaystyle\pdv{\phi}{\rho} \vu{\rho} + \dfrac{1}{\rho} \displaystyle\pdv{\phi}{\varphi} \vu{\varphi} + \displaystyle\pdv{\phi}{z} \vz $ \\
					\hline
						Coordenadas Esféricas & $ \grad{\phi} = \displaystyle\pdv{\phi}{r} \vr + \dfrac{1}{r} \displaystyle\pdv{\phi}{\theta} \vu{\theta} + \dfrac{1}{r\sin{\theta}} \displaystyle\pdv{\phi}{\varphi} \vu{\varphi} $ \\
					\hline
					\hline
				\end{tabular}
			\end{table}
		\item[Divergencia $\div$: ] Mide la diferencia entre el flujo entrante y saliente de un campo vectorial sobre una superficie.
			\begin{table}[H]
				\centering
				\caption{Forma de la divergencia en diferentes sistemas de coordenadas.}
				\label{div}
				\begin{tabular}{||c|c||}
					\hline
					\hline
						Coordenadas Cartesianas & $ \div{\phi} = \displaystyle\pdv{\phi _x}{x} +\displaystyle\pdv{\phi _y}{y} + \displaystyle\pdv{\phi _z}{z} $ \\
					\hline
						Coordenadas Cilíndricas & $ \div{\phi} = \dfrac{1}{\rho}\displaystyle\pdv{(\rho \phi _\rho)}{\rho} + \dfrac{1}{\rho} \displaystyle\pdv{\phi _\varphi}{\varphi} + \displaystyle\pdv{\phi _z}{z} $ \\
					\hline
						Coordenadas Esféricas & $ \div{\phi} = \dfrac{1}{r^2} \displaystyle\pdv{(r^2 \phi _r)}{r} + \dfrac{1}{r\sin{\theta}} \displaystyle\pdv{\qty(\sin{\theta} \phi _\theta)}{\theta} + \dfrac{1}{r\sin{\theta}} \displaystyle\pdv{\phi _\varphi}{\varphi} $ \\
					\hline
					\hline
				\end{tabular}
			\end{table}
		\item[Rotacional $\curl$: ] Muestra la tendencia de un campo a inducir rotación alrededor de un punto.
			$$ \curl{\Ff} = \frac{1}{h_1 h_2 h_3} \left|\begin{array}{ccc}	 
				h_1 \vu{q}_1 & h_2 \vu{q}_2 & h_3 \vu{q}_3 \\
				\pdv{q_1} & \pdv{q_2} & \pdv{q_3} \\
				h_1 F_1 & h_2 F_2 & h_3 F_3 
			\end{array}\right|, $$
			donde $h_i (q_1 ,q_2 ,q_3) = \norm{\vec{e}_i} = \norm{\pdv{\vec{r}}{q_i}}$ es llamado, \textbf{factor de escala} (para coordenadas ortogonales).
		\item[Laplaciano $\laplacian$: ] Existen dos tipos de laplacianos, nosotros nos concentraremos en el laplaciano aplicado a un campo escalar. Esta es una medida del grado en que difiera el campo en un determinado punto en el espacio del valor promedio del campo alrededor de ese punto (vecindad). Maxwell propuso llamar al laplaciano del campo, \textbf{la concentración}.
		\begin{table}[H]
				\centering
				\caption{Forma del Laplaciano (escalar) en diferentes sistemas de coordenadas.}
				\label{lap}
				\begin{tabular}{||c|c||}
					\hline
					\hline
						Coordenadas Cartesianas & $ \laplacian{\phi} = \displaystyle\pdv[2]{\phi}{x} +\displaystyle\pdv[2]{\phi}{y} + \displaystyle\pdv[2]{\phi}{z} $ \\
					\hline
						Coordenadas Cilíndricas & $ \laplacian{\phi} = \dfrac{1}{\rho} \displaystyle\pdv{\rho} \qty(\rho \displaystyle\pdv{\phi}{\rho}) + \dfrac{1}{\rho ^2} \displaystyle\pdv[2]{\phi}{\varphi} + \displaystyle\pdv[2]{\phi}{z} $ \\
					\hline
						Coordenadas Esféricas & $ \laplacian{\phi} = \dfrac{1}{r^2} \displaystyle\pdv[2]{r} \qty(r^2 \displaystyle\pdv{\phi}{r}) + \dfrac{1}{r^2 \sin{\theta}} \displaystyle\pdv{\theta} \qty(\sin{\theta} \displaystyle\pdv{\phi}{\theta}) + \frac{1}{r^2 \sin ^2 {\theta}} \displaystyle\pdv[2]{\phi}{\varphi} $ \\
					\hline
					\hline
				\end{tabular}
			\end{table}
	\end{description}
\end{mdframed}




Teniendo esto en mente, es momento de introducir la ecuación de Laplace.




\begin{mdframed}[style=warning]
	{\Large \textbf{Ecuación de Laplace:}} \\
	Es una ecuación en derivadas parciales de segundo orden de tipo elíptica, es una simplificación de la ecuación de Poisson, así como de la ecuación de Helmholtz.
		$$ \laplacian{u} = 0. $$
\end{mdframed}


Resolviendo esta ecuación en coordenadas esféricas (que es la solución que nos servirá en la parte de aplicaciones en electrostática), por medio de separación de variables, llegamos a
	$$ \psi (r,\theta ,\varphi) = \sum _{l,m} \qty(A_{lm} r^l + B_{lm} r^{-(l + 1)}) P_l ^m (\cos{\theta}) \qty(A_{lm} ^\prime \sin{m\varphi} + B_{lm} ^\prime \cos{m\varphi}). $$
	
Una aplicación ya estudiada por nosotros es la vista en el curso de electromagnetismo. En la cual se estudia una esfera conductora sumergida en un campo eléctrico constante (o variable) estas aplicaciones se dejarán como problemas.







\section*{Polinomios de Hermite}
Al igual que los polinomios de Legendre, los polinomios de Hermite son soluciones (eigenfunciones) de un operador de Sturm-Liouville, además son generados por una función generatriz.

\begin{mdframed}[style=warning]
	{\Large \textbf{Ecuación de Hermite:}} \\
	Dada la ecuación de \textbf{Hermite}
		$$ H_n '' (x) - 2xH_n '(x) + 2nH_n (x) = 0. $$
	Y su "\textbf{Fórmula de Rodríguez}" nace de la función generatriz
		$$ g(x,t) = e^{-t^2 + 2tx} = \sum _{n = 0} ^\infty H_n (x) \frac{t^n}{n!}, $$
	utilizando series de potencias
		$$ H_n (x) = (-1)^n e^{x^2} \dv[n]{x} e^{-x^2}. $$
\end{mdframed}

Estos polinomios aparcen en el oscilador armónico cuántico, movimiento Browniano, análisis de señales, etc.











\section*{Funciones de Bessel (1er Tipo)}

Las funciones de Bessel de 1er orden tienen aplicaciones en ramas como la óptica, electromagnetismo, física de partículas, cuántica, calor y oscilaciones, su capacidad para describir problemas con simetría cilíndrica las hacen útiles en estos contextos. Estas surgen de la ecuación conocida como la ecuación de Bessel\footnote{Se recomienda ver la resolución de dicha ecuación en el libro de Chow, p321-324.}.

\begin{mdframed}[style=warning]
	{\Large \textbf{Ecuación de Bessel:}} \\
		$$ x^2 J_\nu '' + xJ_\nu ' + (x^2 - \nu ^2) J_\nu = 0. $$
		$$ (xy')' - \frac{\nu ^2}{x} y + \lambda xy = 0. $$	
	Función generatriz:
		$$ g(x,t) = e^{(x/2)(t - 1/t)} = \sum _{-\infty} ^\infty J_n (x) t^n , $$
	al igual que las funciónes vistas anteriormente; utilizando series de potencias, en concreto el método de Frobenius
		$$ J_n (x) = \sum _{s = 0} ^\infty \frac{(-1)^s}{s! (n + s)!} \qty(\frac{x}{2})^{n + 2s}, $$
		$$ J_{-n} (x) = \sum _{s = 0} ^\infty \frac{(-1)^s}{s! (s + n)!} \qty(\frac{x}{2})^{n +2s} = (-1)^n J_n (x). $$
	Utilizando la función Gamma\footnote{$\Gamma (z) = \int _0 ^\infty t^{z - 1} e^{-t} \dd{t}$.}
		$$ J_\nu (x) = \sum _{s = 0} ^\infty \frac{(-1)^s}{s! \Gamma (\nu + s + 1)} \qty(\frac{x}{2})^{\nu + 2s}. $$
\end{mdframed}


\begin{mdframed}[style=warning]
	{\Large \textbf{Propiedades de las \href{http://www.sc.ehu.es/sbweb/fisica3/especial/bessel/bessel.html}{Funciones de Bessel:}}} \\
	\begin{enumerate}[a)]
		\item Las funciones de Bessel son pares si $\nu$ es par e impar si $\nu$ es impar.
		\item Las funciones de Bessel están definidas y son continuas para todo $x\in \R$.
		\item Las funciones de Bessel tienen infinitas raíces reales.
		\item Se cumple
		\begin{itemize}
			\item $\dv{x} \qty[x^{-p} J_\nu (x)] = -x^{-p} J_{\nu + 1} (x)$ para $\nu = 0,1,2,\ldots$.
			\item $\dv{x} \qty[x^{p} J_\nu (x)] = x^{p} J_{\nu - 1} (x)$ para $\nu = 0,1,2,\ldots$.
		\end{itemize}
		\item Igualdad asintótica
			$$ J_\nu \approx \sqrt{\frac{2}{\pi x}} \cos{\qty(x - \frac{\nu \pi}{2} - \frac{\pi}{4})}, $$
			y sean $\mu _1$ y $\mu _2$ dos raíces consecutivas cualesquiera de la función de Bessel, entonces $\abs{\mu _1 - \mu -2} \approx \pi$.
		\item Las funciones $J_\nu (\mu _1 x), J_\nu (\mu _2 x), \ldots$ son un sistema completo y cualquier funcion $f\in L^2 (0,b)$ puede escribirse como una serie de Bessel-Fourier.
			$$ f(x) = \sum _{k = 1} ^\infty c_k J_\nu (\mu _k x), $$
		donde $ c_k = \frac{\inner{J_\nu (\mu _k x)}{f(x)}_x}{\norm{J_\nu (\mu _k x)}_x} $.
	\end{enumerate}
\end{mdframed}









\section*{Funciones de Bessel (2do Tipo)}

Dadas las características de las funciones de Bessel de 1er tipo y la ecuación diferencial de la cuál salen, es necesario obtener una segunda solución independiente. A esta segunda solución se le conoce como \textbf{funciones de Bessel de 2do tipo}, funciones de Neumann o de Weber y estan denotadas por $Y_\nu (x)$ o $N_\nu (x)$.


\begin{mdframed}[style=warning]
	{\Large \textbf{Funciones de Bessel de 2do Tipo:}} \\
		$$ Y_\nu (x) = \frac{\cos{(\nu \pi)} J_\nu (x) - J_{-\nu} (x)}{\sin{(\nu \pi)}}, $$
	para $\nu \, \notin \, \Z$.	
\end{mdframed}






\pagebreak


\section*{Problemas}


\begin{ejercicio}
	Encuentre el potencial en todas las regiones de un capacitor esférico (radio $R_o$), con la siguiente condición de frontera.
		$$ \phi (R_o ,\theta) = \left\{ \mqty{V & 0\leq \theta \leq \pi /2 \\ -V & \pi/2 < \theta \leq \pi.} \right. $$
\end{ejercicio}












\begin{ejercicio}
	Demuestre que los polinomios de Hermite son ortogonales y encuentre su norma utilizando la función peso $w(x) = e^{-x^2 /2}$.
\end{ejercicio}














\begin{ejercicio}
	Demuestre las relaciones de recurrencia para las funciones de Hermite:
	\begin{itemize}
		\item $H_{n + 1} (x) = 2xH_n (x) - H_n ' (x)$.
		\item $H_{n + 1} ' (x) = 2(n + 1) H_n (x) = 2H_n (x) + 2xH_n ' (x) - H_n '' (x)$.
	\end{itemize}
\end{ejercicio}













\begin{ejercicio}
	Demuestre las relaciones de recurrencia para las funciones de Bessel de 1er orden:
	\begin{itemize}
		\item $J_{n - 1} (x) + J_{n + 1} (x) = \frac{2n}{x} J_n (x)$.
		\item $J_{n - 1} (x) - J_{n + 1} (x) = 2J_n ' (x)$.
	\end{itemize}
\end{ejercicio}












\begin{ejercicio}
	Realize lo solicitado en el problema 2, pero para las funciones de Bessel. Utilice esta integral para iniciar:
		$$ \int _0 ^1 x J_\nu (\lambda x) J_\nu (\mu x) \dd{x}. $$
\end{ejercicio}




















%%%%