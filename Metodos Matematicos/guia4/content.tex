%\vspace{1cm}
\section*{Introducción}
Luego de estudiar las características, definiciones, teoremas, etc. de los espacios de Banach y Hilbert, es momento de introducir los operadores en dichos espacios, sistemas ortonormales completos con características interesantes. En concreto, se iniciará este estudio por medio de las series de fourier.


\section*{Compacidad}

Antes de estudiar las series de fourier, es bueno y necesario hacer un pequeño estudio sobre los espacios compactos y el concepto de compacidad.


\begin{mdframed}[style=warning]
	{\large \textbf{Espacio Métrico Compacto}} \\
	Un espacio métrico $(X,\rho)$ es compacto si y solo si toda sucesión en $X$ tiene una subsucesión convergente en $X$.
\end{mdframed}

Desde el punto de vista de la computación, es equivalente pensar en la compacidad como si existe una función de búsqueda que pase por todos los puntos del espacio en un tiempo finito. 


\begin{mdframed}[style=warning]
	{\large \textbf{Operador Compacto}} \\
	Un espacio métrico $(X,\rho)$ es compacto si y solo si toda sucesión en $X$ tiene una subsucesión convergente en $X$.
\end{mdframed}



\section*{Operador de Sturm-Liouville}
Si bien, no entraremos al estudio de los operadores de Sturm-Liouville dado que no tenemos toda la teoría que conlleva ello; sin embargo, del operador se deriva algo importante para el tema de estudio de esta guía.



\begin{mdframed}[style=warning]
	{\large \textbf{Operador de Sturm-Liouville}} \\
	Un operador de Sturm-Liouville definido sobre un espacio de funciones con una segunda derivada continua, $y''(t) \in \mathcal{C} _2 (a,b)$, donde $-\infty < a < b < \infty$, opera de la forma
		$$ L y(t) = \qty(p(t) y'(t))' + q(t) y(t) = x(t), $$
	con $x(t)\in \mathcal{C} _2 (a,b)$ si las funciones reales $p(t)$, $p'(t)$ y $q(t)$ son continuas en $[a,b]$.
\end{mdframed}

Es claro que este operador es lineal (aditivo y homogeneo); más adelante se verán ciertas condiciones para que dicho operador sea tanto homogeneo como compacto. Si el operador cumple estas condiciones se le conoce como el \textbf{problema de Sturm-Liouville Regular (PSLR).} A continuación, se mostrarán proposiciones importantes.




\begin{mdframed}[style=warning]
	{\large \textbf{Integral de Fredlhom}} \\
	Sea el operador integral de Fredlhom dado por
		$$ y(t) = Ax(t) = \int K(x,t) x(s) \dd{s}, $$
	donde $K(s,t)$ es una función continua llamada el núcleo del operador integral y cumple la condición
		$$ K = \iint \abs{K(x,t)}^2 < \infty . $$
	Esto es nos permite obtener la siguiente desigualdad si evaluamos la norma de un operador en $L^2$.
		$$ \norm{Lx}^2 = \norm{y}^2 \leq K. $$
\end{mdframed}




\begin{mdframed}[style=warning]
	{\large \textbf{Proposiciones Útiles}}
	\begin{itemize}
		\item Todo PSLR tiene un SOC\footnote{Sistema Ortonormal Completo.} asociado a un operador integral de Fredlhom y toda función $y$ que satisfaga las condiciones de contorno puede ser desarrollada en una expansión de fourier por ese SOC. De forma explícita tenemos la construción
			$$ y(t) = Ax(t) = \sum \lambda _k \inner{e_k}{x} e_k (t) . $$
		\item Los polinomios $\sin{nx}$, $\cos{nx}$ son densos en $L^2$.
		\item Los polinomios $\sin{nx}$, $\cos{nx}$ forman un SOC en $L^2$.
	\end{itemize}
\end{mdframed}

\section*{Series de Fourier}

Las series de Fourier son una herramienta básica en el análisis de Fourier empleado para analizar funciones periódicas a través de la descomposición de dicha función en una suma infinita de funciones sinusoidales (combinación lineal de funciones seno y coseno). Si recordamos bien, en la guía $1$ sabemos que dada una base ortonormal (ahora también completa) se tiene que un elemento cualquiera del espacio se puede escribir de la forma
	$$ f = \sum _{n = 1} ^\infty c_n f_n, $$
con $c_n$ son los llamados \textit{Coeficientes de Fourier} y $f_n = \mqty(\sin{\frac{2n\pi}{T} x} \\ \cos{\frac{2n\pi}{T} x})$ (Notese que no es un conjunto con norma $1$, sin embargo dicha norma depende del periodo $T$ y es $T/2$), donde $T$ es el periodo de la función. 


\begin{mdframed}[style=warning]
	{\large \textbf{Serie de Fourier}} \\
	Sea $f(x)$ una función de variable real con periodo $T$ o integrable en el intervalo $[t_o - T/2,t_o + T/2]$, entonces la serie de Fourier asociada a $f(x)$ es
	\begin{equation}
		f(x) \sim \frac{a_o}{2} + \sum _{n = 1} ^\infty \qty[a_n \cos{\frac{2n\pi}{T} x} + b_n \sin{\frac{2n\pi}{T} x}]. \label{fourier}
	\end{equation}
	Donde
		$$ a_n = \frac{2}{T} \inner{\cos{\frac{2n\pi}{T} x}}{f}, $$
		$$ b_n = \frac{2}{T} \inner{\sin{\frac{2n\pi}{T} x}}{f}, $$
	y
		$$ a_o = \frac{2}{T} \int _{-T/2} ^{T/2} f(x) \dd{x}. $$
	
\end{mdframed}


Dado esto, es normal y lógico pensar si esta representación en realidad converge y, en caso de hacerlo, si converge a $f(x)$. Las condiciones que aseguran la convergencia de dicha serie fueron estudiadas por el matemático Dirichlet.



\begin{mdframed}[style=warning]
	{\large \textbf{Teorema de Dirichlet: Convergencia a una Función Periódica}}
	\begin{enumerate}
		\item $f(x)$ es definida.
		\item $f(x)$ función periódica con periodo $T$.
		\item $f(x)$ y $f'(x)$ son continuas por intervalo en $(-T/2,T/2)$.
	\end{enumerate}
	Entonces $\eqref{fourier}$ converge a:
	\begin{enumerate}
		\item $f(x)$ si $x$ es un punto de continuidad.
		\item $\frac{1}{2} \qty(f(x+) + f(x-))$ si $x$ es un punto de discontinuidad (Con $f(x+) \displaystyle\lim _{t \to x+} f(t)$ y $f(x-) \displaystyle\lim _{t \to x-} f(t)$).
	\end{enumerate}
	
\end{mdframed}






\begin{mdframed}[style=warning]
	{\large \textbf{Identidad de Parseval}} \\
	Para una función en el intervalo $[-\pi ,\pi]$ se tiene la siguiente identidad
		$$ \frac{1}{\pi} \int f(x) ^2 \dd{x} = \frac{a_o ^2}{2} + \sum _{n = 1} ^\infty \qty(a_n ^2 + b_n ^2). $$	
\end{mdframed}



\pagebreak


\section*{Problemas}



\begin{ejercicio}
	Demuestre que la solución de un PSLR es única.
\end{ejercicio}





\begin{ejercicio}
	Demuestre que la Función Delta de Dirac $\delta (x - a)$, expandida en una serie de Fourier en el medio intervalo $(0,L)$ (con $0 < a < L$) está dada por
		$$ \delta (x - a) = \frac{2}{L} \sum _{n = 1} ^\infty \sin{\qty(\frac{n\pi a}{L})} \sin{\qty(\frac{n\pi x}{L})}. $$
\end{ejercicio}





\begin{ejercicio}
	Encuentre la serie de Fourier para la función dada.
	\begin{enumerate}
		\item $ f(x) = \left\{ \mqty{ 0, \qquad -\pi < x < 0 \\ 1, \qquad 0\leq x < \pi } \right. $
		\item $ f(x) = \left\{ \mqty{ 0, \qquad -\pi < x < 0 \\ x^2, \qquad 0\leq x < \pi } \right. $
		\item $ f(x) = \left\{ \mqty{ 0, \qquad -\pi < x < 0 \\ \sin{x}, \qquad 0\leq x < \pi } \right. $
	\end{enumerate}
\end{ejercicio}



\begin{ejercicio}
	Utilice el inciso correspondiente del ejercicio anterior para encontrar el valor de
	\begin{enumerate}
		\item $1 - \dfrac{1}{3} + \dfrac{1}{5} - \dfrac{1}{7} + \dfrac{1}{9} - \cdots$
		\item $1 + \dfrac{1}{2^2} + \dfrac{1}{3^2} + \dfrac{1}{4^2} + \cdots$ y $1 - \dfrac{1}{2^2} + \dfrac{1}{3^2} - \dfrac{1}{4^2} + \cdots$
		\item $\dfrac{1}{2} + \dfrac{1}{1\cdot 3} - \dfrac{1}{3\cdot 5} + \dfrac{1}{5\cdot 7} - \dfrac{1}{7\cdot 9} + \cdots$
	\end{enumerate}
\end{ejercicio}




\begin{ejercicio}
	Demuestre que la serie de Fourier para una onda de sierra ($f(x) = x$ en $-\pi < x < \pi$) es
		$$ f(x) = 2\sum _{n = 1} ^\infty \frac{(-1)^{n + 1}}{n} \sin{nx}. $$
\end{ejercicio}










%%%%