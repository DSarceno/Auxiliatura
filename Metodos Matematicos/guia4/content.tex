%\vspace{1cm}
\section*{Introducción}
Luego de estudiar las características, definiciones, teoremas, etc. de los espacios de Banach y Hilbert, es momento de introducir los operadores en dichos espacios, sistemas ortonormales completos con características interesantes. En concreto, se iniciará este estudio por medio de las series de fourier.


\section*{Compacidad}

Antes de estudiar las series de fourier, es bueno y necesario hacer un pequeño estudio sobre los espacios compactos y el concepto de compacidad.


\begin{mdframed}[style=warning]
	{\large \textbf{Espacio Métrico Compacto}} \\
	Un espacio métrico $(X,\rho)$ es compacto si y solo si toda sucesión en $X$ tiene una subsucesión convergente en $X$.
\end{mdframed}

Desde el punto de vista de la computación, es equivalente pensar en la compacidad como si existe una función de búsqueda que pase por todos los puntos del espacio en un tiempo finito. 


\begin{mdframed}[style=warning]
	{\large \textbf{Operador Compacto}} \\
	Un espacio métrico $(X,\rho)$ es compacto si y solo si toda sucesión en $X$ tiene una subsucesión convergente en $X$.
\end{mdframed}



\section*{Operador de Sturm-Liouville}
Si bien, no entraremos al estudio de los operadores de Sturm-Liouville dado que no tenemos toda la teoría que conlleva ello; sin embargo, del operador se deriva algo importante para el tema de estudio de esta guía.



\begin{mdframed}[style=warning]
	{\large \textbf{Operador de Sturm-Liouville}} \\
	Un operador de Sturm-Liouville definido sobre un espacio de funciones con una segunda derivada continua, $y''(t) \in \mathcal{C} _2 (a,b)$, donde $-\infty < a < b < \infty$, opera de la forma
		$$ L y(t) = \qty(p(t) y'(t))' + q(t) y(t) = x(t), $$
	con $x(t)\in \mathcal{C} _2 (a,b)$ si las funciones reales $p(t)$, $p'(t)$ y $q(t)$ son continuas en $[a,b]$.
\end{mdframed}

Es claro que este operador es lineal (aditivo y homogeneo); más adelante se verán ciertas condiciones para que dicho operador sea tanto homogeneo como compacto. Si el operador cumple estas condiciones se le conoce como el \textbf{problema de Sturm-Liouville Regular (PSLR).} A continuación, se mostrarán proposiciones importantes.




\begin{mdframed}[style=warning]
	{\large \textbf{Integral de Fredlhom}} \\
	Sea el operador integral de Fredlhom dado por
		$$ y(t) = Ax(t) = \int K(x,t) x(s) \dd{s}, $$
	donde $K(s,t)$ es una función continua llamada el núcleo del operador integral y cumple la condición
		$$ K = \iint \abs{K(x,t)}^2 < \infty . $$
	Esto es nos permite obtener la siguiente desigualdad si evaluamos la norma de un operador en $L^2$.
		$$ \norm{Lx}^2 = \norm{y}^2 \leq K. $$
\end{mdframed}




\begin{mdframed}[style=warning]
	{\large \textbf{Proposiciones Útiles}} \\
	\begin{itemize}
		\item Todo PSLR tiene un SOC\footnote{Sistema Ortonormal Completo.} asociado a un operador integral de Fredlhom y toda función $y$ que satisfaga las condiciones de contorno puede ser desarrollada en una expansión de fourier por ese SOC. De forma explícita tenemos la construción
			$$ y(t) = Ax(t) = \sum \lambda _k \inner{e_k}{x} e_k (t) . $$
		\item Los polinomios $\sin{nx}$, $\cos{nx}$ son densos en $L^2$.
		\item Los polinomios $\sin{nx}$, $\cos{nx}$ forman un SOC en $L^2$.
	\end{itemize}
\end{mdframed}

PAGINA 6,7,8 gmet5

\pagebreak


\section*{Problemas}



\begin{ejercicio}
	Demuestre que la solución de un PSLR es única.
\end{ejercicio}



















%%%%