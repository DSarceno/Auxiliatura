\vspace{1cm}


\begin{table}[H]
	\centering
	\begin{tabular}{||c|p{13cm}||c||}
		\hline
		\hline
			No. & Temas & Guía \\
		\hline
		\hline
			1  & Bienvenida &  \\
		\hline
			2  & Introducción, espacios lineales, espacios euclídeos &  \\
			3  & Producto escalar en distintos espacios, desigualdad de Cauchy-Schwarz. & \multirow{4}{2cm}{Guía $1$} \\
			4  & Propiedades métricas de espacios euclídeos, ortogonalidad.  &  \\
			5  & Independencia lineal, base ortonormal,  ortogonalización de Gram-Schmidt. &  \\
		\hline
			6  & Sistemas completos en espacios de dimensión finita, espacios normados, espacios métricos. & \multirow{4}{2cm}{Guía $2$} \\
			7  & Ejemplo de espacio normado, espacios métricos, secuencias. &  \\
			8  & Secuencias, convergencia, límite de secuencias. &  \\
			9  & Secuencias fundamentales. &  \\
		\hline
			10 & Ejemplos sec. fundamentales, definición espacios completos. & \multirow{7}{2cm}{Guía $3$} \\
			11 & Ejemplos secuencias fundamentales, convergentes, espacios completos. &  \\
			12 & Espacios de Banach. &  \\
			13 & Espacio $l^2$, completitud de $l^2$, espacios $l^p$. &  \\
			14 & Espacios $L_p$, espacios de Hilbert. &  \\
			15 & Ejemplo espacio $L_2$, bases ortonormales. &  \\
			16 & Sistemas ortonormales y desigualdad de Bessel. &  \\
		\hline
			17 & Convergencia de series generalizadas de Fourier, sistemas completos. & \multirow{5}{2cm}{Guía $4$} \\
			18 & Teorema de Parseval, sistemas completos y representación en series. & \\
			19 & Ejemplo series de Fourier clásica. & \\
			20 & Convergencia de serie de fourier clásica. & \\
			21 & Convergencia de serie de foudier clásica (continuación). & \\
		\hline
			22 & Series de Fourier Generalizadas. & \multirow{6}{2cm}{Guía $5$} \\
			23 & Formas lineales y operadores lineales. & \\
			24 & Representaciones de operadores lineales en espacios de dimensión finita. & \\
			25 & Norma de operador, operador adjunto. & \\
			26 & Operador adjunto, simétrico, espectro de un operador. & \\
			27 & Espectro de operador, teorema espectral para operador simétrico. & \\
		\hline
			28 & Operadores diferenciales. & \multirow{6}{2cm}{Guía $6$} \\
			29 & Operador diferencial adjunto y autoadjunto. & \\
			30 & Operador de Sturm-Liouville. & \\
			31 & Ejemplo de Ecuación de Sturm-Liouville. & \\
			32 & El problema de Sturm-Liouville. & \\
			33 & Ejemplo problema Sturm-Liouville. & \\
		\hline
		\hline
	\end{tabular}
\end{table}



\begin{table}[H]
	\centering
	\begin{tabular}{||c|p{13cm}||c||}
		\hline
		\hline
			No. & Temas & Guía \\
		\hline
		\hline
			34 & Problemas no homogéneos y el problema singular. & \multirow{1}{2cm}{Guía $7$} \\
		\hline
		\hline
	\end{tabular}
\end{table}





%%%%