%\vspace{1cm}
\section*{Función de Green}
Luego de estudiar los operadores diferenciales lineales, ahora nos enfocaremos en métodos basados en operadores integrales, en concreto, los llamados\textbf{funciones de Green}. Estas funciones nos "desbloquean" la solución a problemas que contengan un término no homogeneo (término fuente), relacionandolo con un operador integral que contenga esta fuente. \\


\href{https://youtu.be/ism2SfZgFJg?si=Mi-57CoReH2CGW91}{VIDEO} \\
\href{https://en.wikipedia.org/wiki/Green\%27s_function#Table_of_Green's_functions}{TABLA DE FUNCIONES} \\
\href{https://brilliant.org/wiki/greens-functions-in-physics/}{FUNCIONES DE GREEN EN LA FÍSICA} \\
\href{https://es.wikipedia.org/wiki/Identidades_de_Green}{identidades}
\begin{mdframed}[style=warning]
	{\Large \textbf{Delta de Dirac:}} \\
	Esta es una herramienta bastante utilizada en electrostática, pero la idea física detrás de ella la pueden leer en \href{https://en.wikipedia.org/wiki/Dirac_delta_function#Motivation_and_overview}{"Motivation and Overview"}. La idea física se puede ver como una bola de billar estática en el tiempo $t = 0$, luego es golpeada por otra bola impartiendo un momento $p$, el intercambio de momentum no es instantaneo, pero para efectos prácticos la energía se transfiere instantáneamente. La fuerza sería $p\cdot \delta (t)$. A esta se le conoce como la función Delta de Dirac
		$$ \delta (x) \simeq \left\{ \mqty{ \infty , & \quad x = 0 ; \\ 0, & \quad x \neq 0. } \right. $$
	Esta función cumple con algunas propiedades utiles:
	\begin{itemize}
		\item $\int _{-\infty} ^\infty \delta (x) \dd{x} = 1.$
		\item $ \int _{-\infty} ^\infty f(t) \delta (t - \tau) \dd{t} = f(\tau). $
	\end{itemize}
\end{mdframed}




Teniendo esto en mente, es momento de introducir la ecuación de Laplace.




\begin{mdframed}[style=warning]
	{\Large \textbf{Función de Green:}} \\
	Para un operador diferencial no homogeneo se tiene a la función de Green como una forma de solución del sistema. Es decir que si $L$ es un operador diferencial lineal, entonces
	\begin{itemize}
		\item La función de Green $G$ es la solución a la ecuación $LG = \delta$, con $\delta$ como la delta de Dirac.
		\item la solución al problema de valores iniciales $Ly = f$ es la convolución de $G\ast f$.
	\end{itemize}
\end{mdframed}









\pagebreak


\section*{Problemas}


\begin{ejercicio}
	Encuentre el potencial en todas las regiones de un capacitor esférico (radio $R_o$), con la siguiente condición de frontera.
		$$ \phi (R_o ,\theta) = \left\{ \mqty{V & 0\leq \theta \leq \pi /2 \\ -V & \pi/2 < \theta \leq \pi.} \right. $$
\end{ejercicio}












\begin{ejercicio}
	Demuestre que los polinomios de Hermite son ortogonales y encuentre su norma utilizando la función peso $w(x) = e^{-x^2 /2}$.
\end{ejercicio}














\begin{ejercicio}
	Demuestre las relaciones de recurrencia para las funciones de Hermite:
	\begin{itemize}
		\item $H_{n + 1} (x) = 2xH_n (x) - H_n ' (x)$.
		\item $H_{n + 1} ' (x) = 2(n + 1) H_n (x) = 2H_n (x) + 2xH_n ' (x) - H_n '' (x)$.
	\end{itemize}
\end{ejercicio}













\begin{ejercicio}
	Demuestre las relaciones de recurrencia para las funciones de Bessel de 1er orden:
	\begin{itemize}
		\item $J_{n - 1} (x) + J_{n + 1} (x) = \frac{2n}{x} J_n (x)$.
		\item $J_{n - 1} (x) - J_{n + 1} (x) = 2J_n ' (x)$.
	\end{itemize}
\end{ejercicio}












\begin{ejercicio}
	Realize lo solicitado en el problema 2, pero para las funciones de Bessel. Utilice esta integral para iniciar:
		$$ \int _0 ^1 x J_\nu (\lambda x) J_\nu (\mu x) \dd{x}. $$
\end{ejercicio}




















%%%%