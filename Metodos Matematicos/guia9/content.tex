%\vspace{1cm}
\section*{Función de Green}
Luego de estudiar los operadores diferenciales lineales, ahora nos enfocaremos en métodos basados en operadores integrales, en concreto, los llamados\textbf{funciones de Green}. Estas funciones nos "desbloquean" la solución a problemas que contengan un término no homogeneo (término fuente), relacionandolo con un operador integral que contenga esta fuente.


\begin{mdframed}[style=warning]
	{\Large \textbf{Operadores Diferenciales:}} \\
	

\end{mdframed}




Teniendo esto en mente, es momento de introducir la ecuación de Laplace.




\begin{mdframed}[style=warning]
	{\Large \textbf{Ecuación de Laplace:}} \\
	Es una ecuación en derivadas parciales de segundo orden de tipo elíptica, es una simplificación de la ecuación de Poisson, así como de la ecuación de Helmholtz.
		$$ \laplacian{u} = 0. $$
\end{mdframed}









\pagebreak


\section*{Problemas}


\begin{ejercicio}
	Encuentre el potencial en todas las regiones de un capacitor esférico (radio $R_o$), con la siguiente condición de frontera.
		$$ \phi (R_o ,\theta) = \left\{ \mqty{V & 0\leq \theta \leq \pi /2 \\ -V & \pi/2 < \theta \leq \pi.} \right. $$
\end{ejercicio}












\begin{ejercicio}
	Demuestre que los polinomios de Hermite son ortogonales y encuentre su norma utilizando la función peso $w(x) = e^{-x^2 /2}$.
\end{ejercicio}














\begin{ejercicio}
	Demuestre las relaciones de recurrencia para las funciones de Hermite:
	\begin{itemize}
		\item $H_{n + 1} (x) = 2xH_n (x) - H_n ' (x)$.
		\item $H_{n + 1} ' (x) = 2(n + 1) H_n (x) = 2H_n (x) + 2xH_n ' (x) - H_n '' (x)$.
	\end{itemize}
\end{ejercicio}













\begin{ejercicio}
	Demuestre las relaciones de recurrencia para las funciones de Bessel de 1er orden:
	\begin{itemize}
		\item $J_{n - 1} (x) + J_{n + 1} (x) = \frac{2n}{x} J_n (x)$.
		\item $J_{n - 1} (x) - J_{n + 1} (x) = 2J_n ' (x)$.
	\end{itemize}
\end{ejercicio}












\begin{ejercicio}
	Realize lo solicitado en el problema 2, pero para las funciones de Bessel. Utilice esta integral para iniciar:
		$$ \int _0 ^1 x J_\nu (\lambda x) J_\nu (\mu x) \dd{x}. $$
\end{ejercicio}




















%%%%