%\vspace{1cm}
\section*{Introducción}
Ya teniendo las propiedades y definiciones que engloban a los operadores, nos enfocaremos en operadores diferenciales y sus relacionados. Esto abrirá las puertas a problemas y teorías interesantes.


\section*{Operador Diferencial}




\begin{mdframed}[style=warning]
	{\Large \textbf{Operador Diferencial:}} \\
	Sea $D$ un operador y $f$ una función en $L_2$, se dice $D$ es un operador diferencial si:
	$$ D:L_2 \to L_2, $$
	$$ f(x) \mapsto \dv{x} f(x). $$
\end{mdframed}

En concreto nosotros trabajaremos con operadores diferenciales de segundo orden (ecuaciones diferenciales de segundo orden) los cuales son problemas de frontera. Entonces, sea $L$ un operador lineal de orden $p$
	$$ L = a_p (x) \dv[p]{x} + a_{p - 1} (x) \dv[p - 1]{x} + \cdots + a_1 (x) \dv{x} + a_o (x). $$
Para una $L$ de segundo orden y $a_2 (x) \neq 0$. Ahora, definimos $B_1 (y) = b_1$ y $B_2 (y) = b_2$ como
	$$ 
		\left\{\begin{array}{c}
			B_1 (y) = \alpha _{11} y(a) + \alpha _{12} y'(a) + \beta _{11} y(b) + \beta _{12} y'(b) = b_1 \\
			B_2 (y) = \alpha _{21} y(a) + \alpha _{22} y'(a) + \beta _{21} y(b) + \beta _{22} y'(b) = b_2.
		\end{array}\right.			
	$$
De lo cual surgen las siguientes condiciones
\begin{itemize}
	\item \textbf{Condiciones de Robbin}
		$$ \alpha _{11} y(a) + \alpha _{12} y'(a) = b_1, $$
		$$ \beta _{21} y(b) + \beta _{22} y'(b) = b_2 . $$
	\item \textbf{Condiciones Periódicas}
		$$ y(a) = y(b) \qquad \text{Dirichlet,} $$
		$$ y'(a) = y'(b) \qquad \text{Neumann.} $$
\end{itemize}





\begin{mdframed}[style=warning]
	{\Large \textbf{Dominio de un Operador:}} \\
	Dado $L$ un operador diferencial en $L_2 (a,b)$, se define su dominio, $D(L)$, como el conjunto de todas las fuciones para las cuales la derivada de mayor orden de $L$ es de cuadrado integrable y satisface las condiciones $B_1 (y) = B_2 (y) = 0$.
\end{mdframed}




	

\section*{Operador Sturm-Liouville}
Tomando el operador $L$ de la sección anterior, se calcula su operador adjunto, el cual es
\begin{equation}	
	L^\dagger = a_2 \dv[2]{x} + (2a_2 ' - a_1 ') \dv{x} + (a_2 '' - a_1 ' + a_o). \label{Ladjoint}
\end{equation}
Y $L$ es hermítico (autoadjunto) si $L = L^\dagger$ y $D(L) = D(L^\dagger)$. En caso de que solo se satisfaga que $L = L^\dagger$, entonces el operador es formalmente autoadjunto. Estas demostraciones se dejan como ejercicio al lector.


\begin{mdframed}[style=warning]
	{\Large \textbf{Operador de Sturm-Liouville:}} \\
	Un operador de Sturm-Liouville definido sobre un espacio de funciones con una segunda derivada continua, donde $-\infty < a < b < \infty$, está definido de la siguiente forma
		$$ L := \dv{t} \qty[p(t) \dv{t}] + q(t), $$
	si $p(a) \neq 0 \neq p(b)$, este operador resulta simétrico.
\end{mdframed}









\section*{Problema de Sturm-Liouville}
 




\pagebreak


\section*{Problemas}



\begin{ejercicio}
	Tomando un operador diferencial $L$ de segundo orden y $u \in D(L)$ y $v\in D(L^\dagger)$. Demuestre la forma adjunta del operador $L$ mostrada en \eqref{Ladjoint}. Al término extra resultante de la integración se le conoce como \textit{concominante bilineal de $u$ y $v$} y es de la forma
		$$ J(u,v) = a_2 (vu' - uv') + (a_1 - a_2')v. $$
	También, encuentre la forma que debe tener el operador $L$ para ser autoadjunto (hermítico) y encuentre la forma de su concominante.
\end{ejercicio}











%%%%