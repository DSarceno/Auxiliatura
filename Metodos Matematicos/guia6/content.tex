%\vspace{1cm}
\section*{Introducción}
Ya teniendo las propiedades y definiciones que engloban a los operadores, nos enfocaremos en operadores diferenciales y sus relacionados. Esto abrirá las puertas a problemas y teorías interesantes.


\section*{Operador Diferencial}




\begin{mdframed}[style=warning]
	{\Large \textbf{Operador Diferencial:}} \\
	Sea $D$ un operador y $f$ una función en $L_2$, se dice $D$ es un operador diferencial si:
	$$ D:L_2 \to L_2, $$
	$$ f(x) \mapsto \dv{x} f(x). $$
\end{mdframed}

En concreto nosotros trabajaremos con operadores diferenciales de segundo orden (ecuaciones diferenciales de segundo orden) los cuales son problemas de frontera. Entonces, sea $L$ un operador lineal de orden $p$
	$$ L = a_p (x) \dv[p]{x} + a_{p - 1} (x) \dv[p - 1]{x} + \cdots + a_1 (x) \dv{x} + a_o (x). $$
Para un $L$ de segundo orden y $a_2 (x) \neq 0$. Ahora, definimos $B_1 (y) = b_1$ y $B_2 (y) = b_2$ como
	$$ 
		\left\{\begin{array}{c}
			B_1 (y) = \alpha _{11} y(a) + \alpha _{12} y'(a) + \beta _{11} y(b) + \beta _{12} y'(b) = b_1 \\
			B_2 (y) = \alpha _{21} y(a) + \alpha _{22} y'(a) + \beta _{21} y(b) + \beta _{22} y'(b) = b_2.
		\end{array}\right.			
	$$
De lo cual surgen las siguientes condiciones
\begin{itemize}
	\item \textbf{Condiciones de Robbin}
		$$ \alpha _{11} y(a) + \alpha _{12} y'(a) = b_1, $$
		$$ \beta _{21} y(b) + \beta _{22} y'(b) = b_2 . $$
	\item \textbf{Condiciones Periódicas}
		$$ y(a) = y(b) \qquad \text{Dirichlet,} $$
		$$ y'(a) = y'(b) \qquad \text{Neumann.} $$
\end{itemize}





\begin{mdframed}[style=warning]
	{\Large \textbf{Dominio de un Operador:}} \\
	Dado $L$ un operador diferencial en $L_2 (a,b)$, se define su dominio, $D(L)$, como el conjunto de todas las fuciones para las cuales la derivada de mayor orden de $L$ es de cuadrado integrable y satisface las condiciones $B_1 (y) = B_2 (y) = 0$.
\end{mdframed}




	

\section*{Operador Sturm-Liouville}
Tomando el operador $L$ de la sección anterior, se calcula su operador adjunto, el cual es
\begin{equation}	
	L^\dagger = a_2 \dv[2]{x} + (2a_2 ' - a_1 ') \dv{x} + (a_2 '' - a_1 ' + a_o). \label{Ladjoint}
\end{equation}
Y $L$ es hermítico (autoadjunto) si $L = L^\dagger$ y $D(L) = D(L^\dagger)$. En caso de que solo se satisfaga que $L = L^\dagger$, entonces el operador es formalmente autoadjunto. Estas demostraciones se dejan como ejercicio al lector.


\begin{mdframed}[style=warning]
	{\Large \textbf{Operador de Sturm-Liouville:}} \\
	Un operador de Sturm-Liouville definido sobre un espacio de funciones con una segunda derivada continua, donde $-\infty < a < b < \infty$, está definido de la siguiente forma
	\begin{equation}
		L := \dv{t} \qty[p(t) \dv{t}] + q(t), \label{opsturm}
	\end{equation}
	si $p(a) \neq 0 \neq p(b)$, este operador resulta simétrico, además de satisfacer las condiciones de frontera homogéneas
		$$ \alpha y(a) + \beta y'(a) = 0, \qquad \gamma y(b) + \delta y'(b) = 0, $$
	con $\alpha ^2 + \beta ^2 \neq 0 \neq \gamma ^2 + \delta ^2$.
\end{mdframed}



\begin{mdframed}[style=warning]
	{\Large \textbf{Operador No Singular:}} \\
	Tomando las características mostradas en la definición anterior se dice que un operador es \textbf{no singular} si la ecuación
		$$ Ly(t) = \vb{0} (t) $$
	no tiene en $D(L)$ soluciones no triviales.
\end{mdframed}



\begin{mdframed}[style=warning]
	{\Large \textbf{Teorema 6.1.:}} \\
	Todo operador de Sturm-Liouville no singular tiene u conjunto ortonormal completo de autofunciones $e_k (t) \in D(L)$. Además, toda función dos veces diferenciable que satisfaga las condiciones de contorno que especifican el dominio del operador, $y(t) \in D(L)$, tiene un desarrollo de Fourier respecto de los autovectores $e_k (t)$ que converge abosluta y uniformemente.
		$$ y(t) = \sum _{k = 1} ^{\infty} c_n e_k (t), \qquad \quad c_n := \frac{\inner{e_k (t)}{y(t)}}{\inner{e_k (t)}{e_k (t)}}. $$
\end{mdframed}



\section*{Problema de Sturm-Liouville}
 
El estudio del problema de Sturm-Liouville parte del operador definido anteriormente, entonces

\begin{mdframed}[style=warning]
	{\Large \textbf{Problema de Sturm-Liouville Regular}} \\
	Dado el operador \eqref{opsturm}, se define la ecuación de Sturm-Liouville cómo
		$$ -\dv{x} \qty[p(x)\dv{y}{x}] + q(x) y = \lambda w(x) y, $$
	donde las funciones $p(x)$ y $w(x)$ son positivas y $q(x)$ es real. La función $w(x)$ se conoce como una función de densidad o función de peso. El valor de $\lambda$ no se especifica; en concreto, el encontrar los valores $\lambda$ para los que exista una solución no trivial de la ecuación que satisfaga las condiciones de frotnera se denomina \textbf{problema de Sturm-Liouville}.
\end{mdframed}

Es claro que los valores de $\lambda$ representan los valores propios del problema S-L y las soluciones son las funciones propias. \\


Entre los problemas de Sturm-Liouville están la ecuación de Bessel, Legendre, Hermite, Laguerre, entre otras. Varias de estas las estudiaremos en este curso. 




\pagebreak


\section*{Problemas}



\begin{ejercicio}
	Tomando un operador diferencial $L$ de segundo orden y $u \in D(L)$ y $v\in D(L^\dagger)$. Demuestre la forma adjunta del operador $L$ mostrada en \eqref{Ladjoint}. Al término extra resultante de la integración se le conoce como \textit{concominante bilineal de $u$ y $v$} y es de la forma
		$$ J(u,v) = a_2 (vu' - uv') + (a_1 - a_2')v. $$
	También, encuentre la forma que debe tener el operador $L$ para ser autoadjunto (hermítico) y encuentre la forma de su concominante.
\end{ejercicio}


\begin{ejercicio}
	Considere una cuerda fija en los puntos $x = 0$ y $x = L$. Resuelva el siguiente S-L
		$$ \dv[2]{\psi (x)}{x} + k^2 \psi (x) = 0. $$
	Además, determine las funciones $p(x),q(x)$ y $w(x)$ así como el valor $\lambda$ en esta ecuación.
\end{ejercicio}









%%%%