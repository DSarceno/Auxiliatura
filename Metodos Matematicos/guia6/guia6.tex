\input{/home/diego/Documents/Licenciatura/LatexBasic/Preamble_general}

%%%%%%%%%%%%%%%%%%%%%%%%%%%%%%%%%%%%%%%%%%%%%%%%%%%%%%%%%%%
\usepackage{fancyhdr}%formato de pagina
\pagestyle{fancy}%colocar la pagina con el formato deseado
\fancyhead{}
\fancyhead[L]{\footnotesize{Métodos Matemáticos para Física}}  
\fancyhead[C]{Guía 6}
\fancyhead[R]{\footnotesize{\thepage}}
%\fancyhead[LO,RE]{Cálculo 3}
%\fancyhead[RO,LE]{\footnotesize{\thepage}}
\fancyfoot{}
%\fancyfoot[L]{Diego Sarceño}
%\fancyfoot[LO,RE]{Diego Sarceño}
%%%%%%%%%%%%%%%%%%%%%%%%%%%%%%%%%%%%%%%%%%%%%%%%%%%%%%%%%%%
%% NUEVA BARRA INFERIOR, NICEEEE :3
\usepackage{fourier-orns}

\renewcommand\footrule{%
\hrulefill
\raisebox{-2.1pt}
{\quad\decosix\quad}%
\hrulefill}
%%%%%%%%%%%%%%%%%%%%%%%%%%%%%%%%%%%%%%%%%%%%%%%%%%%%%%%%%%%
\newcommand{\inner}[2]{\langle #1 , #2 \rangle}
\newcommand{\metric}[2]{\rho(#1,#2)}	
\newcommand{\seque}[2]{\{ #1_{#2} \}}
\newcommand{\E}{\mathbf{E}}
%%%%%%%%%%%%%%%%%%%%%%%%%%%%%%%%%%%%%%%%%%%%%%%%%%%%%%%%%%%
\usetikzlibrary{shadows}

\newmdenv[shadow=true,shadowcolor=black,font=\sffamily,rightmargin=8pt]{shadedbox}
%%%%%%%%%%%%%%%%%%%%%%%%%%%%%%%%%%%%%%%%%%%%%%%%%%%%%%%%%%%
\definecolor{DS_Black}{HTML}{000000}

\begin{document}
\begin{titlepage}
\input{Header_original}

%\noindent \textbf{Instrucciones: } Resuelva cada uno de los siguientes problemas a \LaTeX  o a mano con letra clara y legible, dejando constancia de sus procedimientos. No es necesaria la carátula, únicamente su identificaciónn y las respuestas encerradas en un cuadro.

\section*{Problema 19.08, S}

Dentro de la pared de una casa, una sección de tubería e agua caliente en forma de $L$ consiste en una pieza recta horizontal de $28cm$ de largo, un codo y una pieza recta vertical de $134cm$ de largo. Una trabe y un castillo mantienen fijos los extremos de esa sección de tubería de cobre. Encuentre la magnitudy dirección del desplazamiento del codo cuando hay flujo de agua, lo que eleva la temperatura de la tubería de $18^o C$ a $46.5^o C$.



\begin{figure}[H]
	\centering
	\includegraphics[scale=0.5]{./img/tubo.png}
	\caption{P19.08S}
\end{figure}



%%%%


\pagebreak


\begin{thebibliography}{00}
\bibitem{b1} Falomir, H. (2015). \textit{Curso de métodos de la física matemática.} Series: Libros de Cátedra.
\bibitem{b2} Arfken, G. B., \& Weber, H. J. (2013). Mathematical methods for physicists.
\end{thebibliography}



\end{titlepage}
\end{document}
