\input{/home/diego/Documents/Licenciatura/LatexBasic/Preamble_general}

%%%%%%%%%%%%%%%%%%%%%%%%%%%%%%%%%%%%%%%%%%%%%%%%%%%%%%%%%%%
\usepackage{fancyhdr}%formato de pagina
\pagestyle{fancy}%colocar la pagina con el formato deseado
\fancyhead{}
\fancyhead[L]{\footnotesize{Métodos Matemáticos para Física}}  
\fancyhead[C]{Guía 2}
\fancyhead[R]{\footnotesize{\thepage}}
%\fancyhead[LO,RE]{Cálculo 3}
%\fancyhead[RO,LE]{\footnotesize{\thepage}}
\fancyfoot{}
%\fancyfoot[L]{Diego Sarceño}
%\fancyfoot[LO,RE]{Diego Sarceño}
%%%%%%%%%%%%%%%%%%%%%%%%%%%%%%%%%%%%%%%%%%%%%%%%%%%%%%%%%%%
%% NUEVA BARRA INFERIOR, NICEEEE :3
\usepackage{fourier-orns}

\renewcommand\footrule{%
\hrulefill
\raisebox{-2.1pt}
{\quad\decosix\quad}%
\hrulefill}
%%%%%%%%%%%%%%%%%%%%%%%%%%%%%%%%%%%%%%%%%%%%%%%%%%%%%%%%%%%
\newcommand{\inner}[2]{\langle #1 , #2 \rangle}
\newcommand{\metric}[2]{\rho(#1,#2)}	
\newcommand{\seque}[2]{\{ #1_{#2} \}}
%%%%%%%%%%%%%%%%%%%%%%%%%%%%%%%%%%%%%%%%%%%%%%%%%%%%%%%%%%%

\begin{document}
\begin{titlepage}
% AUTOR: Diego Sarceño

% ENCABEZADO DE TRABAJOS CON LOGO DE LA UNIDAD ACADÉMICA

% ENCABEZADO LOGO COLOR
%\begin{tabulary}{20cm}{Lp{0.9cm}p{6.1cm}}
%Universidad de San Carlos de Guatemala & & \multirow{4}{8cm}{\hfill %\includegraphics[scale=0.5]{ECFM.png}}\\            % Logo de la unidad academica
%Escuela de Ciencias Físicas y Matemáticas & \hfill & \\
%Diego Sarceño 201900109 & \hfill & \\
%Análisis de Variable Compleja 1 & \hfill & \\
%\today & & \\
%\end{tabulary}\\[0.25cm]


% ENCABEZADO LOGOS
\begin{tabulary}{20cm}{LLCRR}
\multirow{4}{2.3cm}{\includegraphics[scale=0.13]{/home/diego/Documents/Licenciatura/LatexBasic/ECFM.pdf}} & Universidad de San Carlos de Guatemala  & & & \multirow{4}{4.5cm}{\hfill \includegraphics[scale=0.082]{/home/diego/Documents/Licenciatura/LatexBasic/USAC.pdf}}\tabularnewline
 & Escuela de Ciencias Físicas y Matemáticas & \hfill &  & \tabularnewline
 & Física 1 & \hfill ~~ &   & \tabularnewline
 & Auxiliar: Diego Sarceño & &  & \tabularnewline
 & \today &  & & \tabularnewline
\end{tabulary}\\[0.75cm]

{\hrule height 1.5pt} \vspace{0.1cm}
\begin{tabulary}{21cm}{p{4cm}p{11cm}p{4cm}}
    \hfill & \huge{\scshape{Hoja de Trabajo 1}} & \hfill
\end{tabulary}
{\hrule height 1.5pt} 
\vspace{0.5cm}




%\noindent \textbf{Instrucciones: } Resuelva cada uno de los siguientes problemas a \LaTeX  o a mano con letra clara y legible, dejando constancia de sus procedimientos. No es necesaria la carátula, únicamente su identificaciónn y las respuestas encerradas en un cuadro.

\section*{Problema 12.34, Z}



Corre agua hacia una fuente, llenando todos los tubos con una rapidez constante de $0.75 m^3 /s$. \textit{a)} ¿Qué tanto saldrá por un agujero de $4.5cm$ de diámetro? \textit{b)} ¿Con qué rapidez saldrá si el diámetro del agujero es tres veces más grande?




%%%%





\begin{thebibliography}{00}
\bibitem{b1} Bartle, R. G., \& Sherbert, D. R. (2000). Introduction to real analysis. John Wiley \& Sons, Inc..
\bibitem{b2} Rudin, W., Sánchez, M. I. A., \& Aguirre, L. B. (1980). Principios de análisis matemático (Vol. 3). McGraw-Hill.
\end{thebibliography}



\end{titlepage}
\end{document}
